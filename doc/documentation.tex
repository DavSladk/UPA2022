% Copyright 2022 by Marek Rychly <rychly@fit.vut.cz>.
%
\documentclass[10pt,xcolor=pdflatex,dvipsnames,table,oneside]{book}
% babel and encoding
\usepackage[czech]{babel}
\usepackage[T1]{fontenc}
\usepackage[utf8]{inputenc}
\usepackage{lmodern}

\usepackage{csquotes}% correct/formal language-specific quotations
\usepackage{microtype}% character protrusion, font expansion, adjustment of interword spacing, additional kerning, tracking, etc.
\usepackage{hyperref}% hyper-refs in PDF
\usepackage{graphicx}
\usepackage{multirow}

\author{
    František Koleček, \href{mailto:xkolec08@stud.fit.vut.cz}{xkolec08@stud.fit.vut.cz} \\
    Tomáš Moravčík, \href{mailto:xmorav41@stud.fit.vut.cz}{xmorav41@stud.fit.vut.cz} \\
    David Sladký, \href{mailto:xsladk07@stud.fit.vut.cz}{xsladk07@stud.fit.vut.cz}
    }
\title{Ukládání rozsáhlých dat v NoSQL databázích}
\date{zima 2022}

\begin{document}

\pagenumbering{roman}

\hypersetup{pageanchor=false}% disable hyperref anchor to title page as maketitle enforce pagenumbering to arabic which colides the titlepage with the first arabic page below
\maketitle
\hypersetup{pageanchor=true}

\tableofcontents

\newpage% force page-break to start the page numbering on a new page
\pagenumbering{arabic}

\part{Analýza zdrojových dat a návrh jejich uložení v NoSQL databázi}

\chapter{Analýza zdrojových dat}

Použitá datová sada se nachází na stránkách \href{https://mdcr.cz/Dokumenty/Verejna-doprava/Jizdni-rady,-kalendare-pro-jizdni-rady,-metodi-%281%29/Jizdni-rady-verejne-dopravy}{ministerstva dopravy}
a to konrétně \href{https://portal.cisjr.cz/pub/draha/celostatni/szdc/}{zde}. Detailní popis formátu dokumentů datové sady lze najít na stejném \href{https://portal.cisjr.cz/pub/draha/celostatni/szdc/Popis%20DJ%C5%98_CIS_v1_09.pdf}{místě}.

Datová sada se skládá z \verb|XML| souborů. Tyto soubory jsou zveřejňovaný na začátku roku pro celý rok ve složce \verb|GVD|. Dále je pak každý měsíc zveřejňována datová sada pro daný měsíc s aktualizacemi pro spoje.
Každý soubor má element \verb|CZPTTCreation|, který určuje jeho vytvoření.

\verb|XML| soubory lze rozdělit do následujících tří skupin:
\begin{itemize}
    \item Definující spoj
    \item Rušící spoj
    \item Definující náhradní spoj
\end{itemize}
Soubory definující spoje mají jako kořenový element \verb|CZPTTCISMEssage|. První důležité informace se nachází v elementu \verb|Identifiers|,
kde jsou uvedeny identifikátory pro definované spojení a vlak, který ho bude provádět. Dále element \verb|CZPTTHeader| určuje, zda spoj
přijíždí nebo pokračuje z/do zahraniční stanice. Elementy \verb|CZPTTLocation| obsahují jednotlivé stanice, kterýma vlak projiždí. Zde jsou
uvedeny i další informace ke stanici. Nejvýznamnější z nich jsou: čas příjezdu/odjezdu, typ aktivity. Po uvedení všech stanic následuje
element \verb|PlannedCalendar|, který určuje výčet dní, kdy je tento spoj prováděň.

Soubory rušící spoje mají kořenový element \verb|CZCanceledPTTMessage|. Podobně jako soubory definující spoje obsahují identifikaci spoje, který se ruší
a výčet dní, kdy se ruší. Dále už nenesou žádné informace.

Soubory definující náhradní spoje mají stejnou strukturu jako soubory definující spoje jenom s jediným rozdíle. Obsahují element
\verb|RelatedPlannedTransportIdentifiers|, který určuje, jaký spoj nahrazují. Tyto spoje mají unikátní indentifikátor vůči
normálním spojům.

\chapter{Návrh způsobu uložení dat}
Z datové sady poskytnuté XML soubory lze definovat několik entit a vazeb mezi nimi vhodných pro uložení do NoSQL databáze.
\section{File}
Samotné soubory lze považovat za entity pro uložení. Tyto entity ponesou informace o jméně souboru,
který reprezentují a času, kdy byl soubor vytvořeny z elementu CZPTTCreation.

\vspace{1em}
\begin{tabular}{|l|}
    \hline
    :File \\
    \hline
    filename \\
    creation \\
    \hline
\end{tabular}

\subsection{PRECEDES}
Novější soubory aktualizují stav definovaný předcházejícími soubory, proto mezi jednotlivými entitami :File
budou vazby :PRECEDES, které spojí soubor s jeho nejbližším následníkem.

\vspace{1em}
\begin{tabular}{|l|}
    \hline
    :File \\
    \hline
\end{tabular}
\begin{tabular}{c}
    -- :PRECEDES --> \\
\end{tabular}
\begin{tabular}{|l|}
    \hline
    :File \\
    \hline
\end{tabular}

\subsection{DEFINES}
Pokud se nejedná o soubor rušící spoj, tak tento soubor definuje o jakém spoji podává informace. Tuto vlastnost bude modelována vazbou :DEFINES.

\vspace{1em}
\begin{tabular}{|l|}
    \hline
    :File \\
    \hline
\end{tabular}
\begin{tabular}{c}
    -- :DEFINES --> \\
\end{tabular}
\begin{tabular}{|l|}
    \hline
    :PA \\
    \hline
\end{tabular}

\subsection{CANCELS}
Pokud se jedná o soubor rušící spoj, tak tento soubor definuje jakého spoje ruší jízdy. Tuto vlastnost bude modelována vazbou :CANCELS.

\vspace{1em}
\begin{tabular}{|l|}
    \hline
    :File \\
    \hline
\end{tabular}
\begin{tabular}{c}
    -- :CANCELS --> \\
\end{tabular}
\begin{tabular}{|l|}
    \hline
    :PA \\
    \hline
\end{tabular}

\section{PA}
Další významnou entitou jsou spoje. Tyto spoje budou označovány jako PA -- stejně jako v souborech v elementu Object Type.

\vspace{1em}
\begin{tabular}{|l|}
    \hline
    :PA \\
    \hline
    Core \\
    Company \\
    Variant \\
    TimetableYear \\
    NetworkSpecificParameters \\
    Header \\
    \hline
\end{tabular}

\subsection{SERVER BY}
Každý spoje je prováděn vlakem. Tento vlak je vždy uveden v souboru se spojem, který provádí. Tuto vlastnost
definujeme jako :SERVED BY.

\vspace{1em}
\begin{tabular}{|l|}
    \hline
    :PA \\
    \hline
\end{tabular}
\begin{tabular}{c}
    -- :SERVED BY --> \\
\end{tabular}
\begin{tabular}{|l|}
    \hline
    :TR \\
    \hline
\end{tabular}

\subsection{RELATED}
Pokud se jedná o spoj, který nahrazuje jiný spoj, tak je také definováno jaký spoj je nahrazován. Tuto vlastnost
modelujeme jako :RELATED.

\vspace{1em}
\begin{tabular}{|l|}
    \hline
    :PA \\
    \hline
\end{tabular}
\begin{tabular}{c}
    -- :RELATED --> \\
\end{tabular}
\begin{tabular}{|l|}
    \hline
    :PA \\
    \hline
\end{tabular}

\subsection{GOES IN}
U každého spoje je také uvedeno, v jaké dny jezdí. Informaci v jaké dny jeden reprezentujeme vazbou :GOES IN mezi spojem a dnem.

\vspace{1em}
\begin{tabular}{|l|}
    \hline
    :PA \\
    \hline
\end{tabular}
\begin{tabular}{c}
    -- :GOES IN --> \\
\end{tabular}
\begin{tabular}{|l|}
    \hline
    :Day \\
    \hline
\end{tabular}

\subsection{IS IN}
Každý spoj projíždí minimálně dvěmi stanicemi. Tuto vlastnost modelujeme vztahem :IS IN mezi spojem a stanicí.
Tato vazbna také neve řadu parametrů definující příjez, odjezd, prováděné akce a podobné.

\vspace{1em}
\begin{tabular}{|l|}
    \hline
    :PA \\
    \hline
\end{tabular}
\begin{tabular}{c}
    -- :IS IN --> \\
\end{tabular}
\begin{tabular}{|l|}
    \hline
    :Station \\
    \hline
\end{tabular}

\vspace{1em}
\begin{tabular}{|l|}
    \hline
    :IS IN \\
    \hline
    LocationSubsidiaryCode \\
    AllocationCompany \\
    LocationSubsidiaryName \\
    ALA \\
    ALAOffest \\
    ALD \\
    ALDOffset \\
    DwellTime \\
    ResponsibleRU \\
    ResponsibleIM \\
    TrainType \\
    TrafficType \\
    CommercialTrafficType \\
    TrainAvtivityType \\
    OperationalTrainNumber \\
    NetworkSpecificParameters \\
    \hline
\end{tabular}

\section{TR}
Soubory jsou také definovány vlaky, které provádí spoje. Tyto vlaky budou znázorněny entitami :TR.

\vspace{1em}
\begin{tabular}{|l|}
    \hline
    :TR \\
    \hline
    Core \\
    Company \\
    Variant \\
    TimetableYear \\
    NetworkSpecificParameters \\
    \hline
\end{tabular}

\section{Station}
V XML souborech jsou také definováné stanice, kterými vlak projíždí. Tyto stanice budou modelovány pomocí
entit :Station.

\vspace{1em}
\begin{tabular}{|l|}
    \hline
    :Station \\
    \hline
    LocationPrimaryNumber \\
    PrimaryLocationName \\
    CountryCodeISO \\
    \hline
\end{tabular}

\vspace{2em}
\section{Day}
U každého spoje je také určeno, v jaké dny jezdí. Tyto dny budou existovat jeko entity :Day.

\vspace{1em}
\begin{tabular}{|l|}
    \hline
    :Day \\
    \hline
    Date \\
    \hline
\end{tabular}

\begin{figure}[h]
    \includegraphics[width=1\textwidth, angle=0]{img/structure.pdf}
    \caption{Finální schéma entit a vazeb meiz nimi v databázi.}
\end{figure}

\chapter{Zvolená NoSQL databáze}
Pro řešení tohoto problému jsme zvolili grafovou databázi, konkrétně \href{https://neo4j.com/}{Neo4j}. Grafová databáze nám umožní
reprezentovat entity jako uzly v grafu a vztahu mezi entitaby jako cesty mezi těmito uzly. Zároveň je možné u uzlů i
cest definovat další vlastnosti typu klíč:hodnota, co zase umožní uložit veškeré další potřebné informace
vztahující se k jednotlivým objektům. Neo4j poskytuje webové rozhraní pro databázi, které umožňuje jednoduché
zadávání požadavků na databázi. S kombinací s tím, že toto webové rozhraní také zobrazuje získané výsledky do
grafu, tak nám to umožní snadnější opravování chyb v aplikaci a snadnější vytváření potřebných požadavků na databázi.

\part{Návrh, implemetace a použití aplikace}

\chapter{Návrh aplikace}
\iffalse
\paragraph{Cíl:}
Navrhnout hlavní části aplikace splňující požadavky zadání s důrazem na práci s na ni napojenou databází NoSQL či datovými zdroji
(při jejich předzpracování a nahrávání do NoSQL databáze).

\paragraph{Obsah:}
Použité technologie (např. skriptovací jazyk, knihovny, atp.)
a architektura (např. skript či sekvence skriptů pravidelně spouštěných v daných časových intervalech či v reakci na danou událost).
Způsob technického řešení úloh ze zadání (jejich průběh v aplikaci) a konceptů z předchozího návrhu (struktury, algoritmy, toky dat, atp.).

\paragraph{Prostředky:}
Strukturovaný text (sekce, odstavce, odrážky, atp.), případně pseudokód či obrázky, doplňující technické detaily konceptů nastíněných v předchozím návrhu.
Důraz je kladen na způsob realizace dotazů ze zadání.

\paragraph{Fáze projektu:}
Návrh aplikace a částečně po či souběžně s návrhem databáze.
\fi
\section{Zpracování souborů XML}

Informace o vlakových spojích jsou získávány ze souborů ve formátu XML. Tyto soubory je třeba zpracovat – nahrát data do strukturované vnitřní reprezentace programu, pro efektivní nahrávání do databáze. Každý soubor obsahuje informace o jednom konkrétním vlakovém spoji, jeho cesta a časy ve stanicích jsou vždy stejné, jsou zde definované dny, ve kterých tento spoj jede. Nejdůležitější zpracovávaná data jsou:

\begin{itemize}
    \item Identifikátory
    \item Navštívené stanice a časy příjezdu a odjezdu
    \item Dny, ve kterých spoj jede
\end{itemize}

\subsection{Struktura XML souborů}
Každá s těchto informací se nachází ve vlastní „větvi“ v souboru. Nachází se u nich samozřejmě i dodatečné informace – detaily lokace, činnost vlaku ve stanici atd. Platné dny jsou určeny pomocí dvou atributů – začátek a konec platnosti a bitmapa. Bitmapou je myšlen řetězec jedniček a nul, kde jedničky vyjadřují platnost v jednotlivých dnech. Tímto způsobem lze vyjádřit platné dny na rok dopředu pomocí řetězce dlouhém 365 znaků.

\subsection{Zvolené Prostředky}
Zpracovávání souborů je implementováno v jazyce Python v souboru \verb|parser.py|. Je využito modulu \verb|xml|, který je součástí základní instalace Pythonu, není třeba jej dodatečně instalovat. Tento modul obsahuje třídu \verb|ElementTree|, která umožňuje snadné nahrávání dat ze stromové struktury souboru. Velmi užitečnou funkcí je možnost adresování uzlů pomocí „cesty“ - obdobným způsobem jako adresování souborů v souborovém systému.

\section{Sťahovanie dát}
Súbory sú stiahnuté prostredníctvom súboru \verb|download.py| v jazyku Python. Bola využitá knihovna \verb|BeautifulSoup4|, ktorá slúži na získavanie dát z HTML súborov. Ďalším podstatným modulom je \verb|requests|, využitý na volanie GET požiadavkov na stiahnutie \verb|zip| súborov. Súbor \verb|download.py| obsahuje dve triedy; \verb|DataDownload| a \verb|DataExtract|. Pomocou prvej triedy sa stiahnu všetky doposiaľ nestiahnuté súbory náležiace danému roku jízdnich radov a uložia sa do priečinku \verb|/zips|. Druhou triedou sú tieto súbory extrahované už v podobe \verb|xml| súborov, pripravených na ďalšie spracovanie.

\subsection{Způsob implementace}
Nejdůležitější roli při zpracovávání hraje funkce \verb|node_to_dict|, která rekurzivně prohledává daný uzel a převádí jeho obsah na slovníkový datový typ.

Protože platné dny jsou v naší databázi ukládány jako vlastní uzly, na které pak odkazují spoje, které jsou platné v daný den, je potřeba původní reprezentaci platných dní (popsána výše) převést na seznam konkrétních kalendářních dat. Pro implementaci této funkcionality byl využit modul \verb|datetime|, který je opět součástí základní instalace pythonu. Tento modul mimo jiné umožňuje vykonávat „aritmetické operace“ s kalendářními daty. V našem případě se jedná o sečtení data začátku platnosti a indexu jedničky v příslušné bitmapě. Implementace je ve funkci \verb|cal_to_listofdays|.

\section{Práce s databází}
Obsluhování databáze je implementováno v modulu \verb|database.py|. Obsahem tohoto module je třída \verb|Database| s metodami
obsahující query na spravování entit (File, PA, TR, Day, Station), tak vazby mezi nimi (PRECEDES, DEFINES, CANCELS, GOES IN, IS IN, RELATED).
Stěžejní query je však pro vyhledávání spoje:
\begin{verbatim}
MATCH (s:Station)<-[i:IS_IN]-(p:PA)-[ii:IS_IN]->(ss:Station)
MATCH (p)-[:GOES_IN]->(d:Day)
MATCH (sx:Station)<-[ix:IS_IN]-(p)-[ii:IS_IN]->(ss:Station)
WHERE s.PrimaryLocationName="Mosty u Jablunkova"
AND   ss.PrimaryLocationName="Ostrava hl.n."
AND   time(i.ALA) > time("00:00:00")
AND   time(ii.ALA) > time(i.ALA)
AND   d.Date="2022-03-31"
AND   time(ix.ALA) > time(i.ALA)
AND   time(ix.ALA) < time(ii.ALA)
AND   '0001' IN i.TrainActivityType AND i.TrainType='1'
AND   '0001' IN ii.TrainActivityType AND ii.TrainType='1'
RETURN s.PrimaryLocationName, i.ALA, sx.PrimaryLocationName, 
       ix.ALA, ss.PrimaryLocationName, MIN(ii.ALA)
\end{verbatim}

\chapter{Způsob použití}

\section{Neo4j}

Pro spuštění je potřeba mít přístup k Neo4j databázi. Toho se dá dosáhnout několika způsoby, zde však budou uvedeny
jenom dva nejpřímočařejší. Je také potřeba mít v Pythonu doinstalovaný modul Neo4j: \verb|pip install neo4j|.
Pro správnou funkčnost tohoto module je potřeba Python alespoň verze 3.6.

\subsection{Databáze na cloudu}
Pro tuto variantu je potřeba si založit účet na stránkách \href{https://neo4j.com/}{Neo4j} -> Get Started -> Start Free -> Sign up.
Po vytvoření účtu je možné se dostat ke správci cloudových databází obdobným způsoběm opět z \href{https://neo4j.com/}{Neo4j} ->
Get Started -> Start Free -> Přihlášení. Po přihlášení lze vytvořit novou databázi pomocí New Instance. Nezapomeňte si
poznamenat přihlašovací údaje do databáze.

\subsection{Lokální databáze}
Pro lokální databázi je třeba stáhnout \href{https://neo4j.com/download-center/#community}{Neo4j Community}.
Po rozbalení staženého archivu zadejte následující příkaz z kořenového adresáře: \verb|./bin/neo4j console|.
Databáze bude poskytovat webové rozhraní na http://localhost:7474/. Výchozí login i heslo jsou \verb|neo4j|.

\section{Spuštění}
\begin{verbatim}
python3 app.py login heslo uri
\end{verbatim}
\begin{itemize}
    \item \verb|login| - přihlašovací jméno k databázi
    \item \verb|heslo| - heslo k databázi
    \item \verb|uri| - uri databáze
\end{itemize}
Po spuštení programu se bude program dotazovat na několik informací:
\begin{enumerate}
    \item \verb|Do you want to download data? Y/N| - Tato volba způsobí stažení zazipovaných xml do
    složky zips a jejich následovné rozbalení do složky \verb|xml_data|. Pokud nechcete stahovat data,
    umístěte vlastní xml soubory do \verb|xml_data|.
    \item \verb|Do you want to load local data? Y/N| - Projde soubory v \verb|xml_data| a nahraje jejich grafovou
    reprezentaci do databáze.
    \item \verb|Please put starting station| - Zadejte výchozí stanici.
    \item \verb|Please put terminal station| - Zadejte konečnou stanici.
    \item \verb|Please put since date| - Zadejte čas od kdy má program vyhledávat ve formátu \verb|YYYY-MM-DDTHH:MM:SS|. \verb|T| je oddělovač data a času.
\end{enumerate}
Po zadání všech informací program vyhledá nejbližší přímí spoj, pokud takový existuje, mezi stanicemi.
Posléze budete dotázání, zda chcete provést další vyhledávání.

\vspace{1em}
Příklad výstupu programu.
\begin{verbatim}
user@pc:/path/to/root/folder$ python3 app.py neo4j neo4j bolt://localhost:7687
Do you want to download data? Y/N
n
Do you want to load local data? Y/N
n
Please put starting station:
Polanka nad Odrou
Please put terminal station:
Sedlnice
Please put since date:
2022-07-13T9:00:00   
---------------------------------------
Polanka nad Odrou -- 09:43:00
Jistebník -- 09:46:00
Studénka -- 09:51:00
Sedlnice-Bartošovice -- 09:56:00
Sedlnice -- 09:57:00
It took 0.004521846771240234 s to complete this query.
---------------------------------------
Do you wish to continue?
n    
\end{verbatim}

\chapter{Experimenty}
Experimenty byly prováděny na přístroji s tímto HW/SW:
\begin{itemize}
    \item OS: Pos os 22.04
    \item CPU: Ryzen 5 5600X
    \item RAM: 16 GB
\end{itemize}
Experimenty byly prováděny vůči lokálně běžící databázi.

\section{Inicializace databáze}

\vspace{1em}
\begin{tabular}{|l|l|}
    \hline
    Počet spracovaných souborů & 13 051 \\
    \hline
    Doba provedení &  22,9 h \\
    \hline
    Vytvořených GOES IN vazeb & 5 781 132 \\
    \hline
    Vytvořených IS IN vazeb & 552 858 \\
    \hline
\end{tabular}

\section{Dotazování}

\vspace{1em}
\begin{tabular}{|l|l|}
    \hline
    Výchozí stanice & Ropice zastávka \\
    \hline
    Cílová stanice & Dolní Lutyně \\
    \hline
    Datetime & 2022-03-31T00:00:00 \\
    \hline
    Doba provedení & 1.1645660400390625s \\
    \hline
    \multirow{12}{*}{Výsledek} & Ropice zastávka -- 17:36:00 \\
     & Český Těšín nákl.n. -- 17:38:00 \\
     & Český Těšín -- 17:39:00 \\
     & 2.TK loucká -- 17:44:30 \\
     & Chotěbuz (Kocobędz) -- 17:45:00 \\
     & Louky nad Olší -- 17:47:30 \\
     & Karviná-Darkov -- 17:55:00 \\
     & Stan. Karviná-Darkov -- 17:55:30 \\
     & Karviná hl. n. -- 17:58:00 \\
     & Koukolná odbočka -- 18:01:00 \\
     & Dětmarovice -- 18:03:00 \\
     & Dolní Lutyně -- 18:06:30 \\
    \hline
\end{tabular}

\vspace{1em}
\begin{tabular}{|l|l|}
    \hline
    Výchozí stanice & Mosty u Jablunkova \\
    \hline
    Cílová stanice & Ostrava hl.n. \\
    \hline
    Datetime & 2022-03-31T00:00:00 \\
    \hline
    Doba provedení & 1.1052453517913818 \\
    \hline
    Výsledek & No connection found. \\
    \hline
\end{tabular}

\vspace{1em}
\begin{tabular}{|l|l|}
    \hline
    Výchozí stanice & Mosty u Jablunkova \\
    \hline
    Cílová stanice & Ostrava hl.n. \\
    \hline
    Datetime & 2022-03-31T00:00:00 \\
    \hline
    Doba provedení & 1.1052453517913818 \\
    \hline
    Výsledek & No connection found. \\
    \hline
\end{tabular}

\vspace{1em}
\begin{tabular}{|l|l|}
    \hline
    Výchozí stanice & Rajhrad \\
    \hline
    Cílová stanice & Rajhrad \\
    \hline
    Datetime & 2022-06-06T00:00:00 \\
    \hline
    Doba provedení & 0.6875042915344238s \\
    \hline
    Výsledek & No connection found. \\
    \hline
\end{tabular}

\vspace{1em}
\begin{tabular}{|l|l|}
    \hline
    Výchozí stanice & Rajhrad \\
    \hline
    Cílová stanice & Brno-Královo Pole \\
    \hline
    Datetime & 2022-10-24T00:00:00 \\
    \hline
    Doba provedení & 0.01873946189880371s \\
    \hline
    Výsledek & No connection found. \\
    \hline
\end{tabular}

\vspace{1em}
\begin{tabular}{|l|l|}
    \hline
    Výchozí stanice & Rajhrad \\
    \hline
    Cílová stanice & Brno hl.n. \\
    \hline
    Datetime & 2022-24-10T00:00:00 \\
    \hline
    Doba provedení & 0.23762035369873047s \\
    \hline
    Výsledek & No connection found. \\
    \hline
\end{tabular}

\section{Rychlost zpracování XML souborů}
Funkčnost a efektivita skriptu \verb|parser.py| byla testována nad složkou \verb|GVD2022|, která obsahuje přibližně 12300 souborů formátu XML. Všechny soubory byly v experimentu zpracovány a výpis důležitých informací vytisknut na standardní výstup. Tato operace trvala 93 sekund, tedy cca 132 zpracovaných souborů za sekundu.

\end{document}
