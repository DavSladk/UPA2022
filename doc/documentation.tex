% Copyright 2022 by Marek Rychly <rychly@fit.vut.cz>.
%
\documentclass[10pt,xcolor=pdflatex,dvipsnames,table,oneside]{book}
% babel and encoding
\usepackage[czech]{babel}
\usepackage[T1]{fontenc}
\usepackage[utf8]{inputenc}

\usepackage{csquotes}% correct/formal language-specific quotations
\usepackage{microtype}% character protrusion, font expansion, adjustment of interword spacing, additional kerning, tracking, etc.
\usepackage{hyperref}% hyper-refs in PDF

\author{
    František Koleček, \href{mailto:xkolec08@stud.fit.vut.cz}{xkolec08@stud.fit.vut.cz} \\
    Tomáš Moravčík, \href{mailto:xmorav41@stud.fit.vut.cz}{xmorav41@stud.fit.vut.cz} \\
    David Sladký, \href{mailto:xsladk07@stud.fit.vut.cz}{xsladk07@stud.fit.vut.cz}
    }
\title{Ukládání rozsáhlých dat v NoSQL databázích}
\date{zima 2022}

\begin{document}

\pagenumbering{roman}

\hypersetup{pageanchor=false}% disable hyperref anchor to title page as maketitle enforce pagenumbering to arabic which colides the titlepage with the first arabic page below
\maketitle
\hypersetup{pageanchor=true}

\tableofcontents

\newpage% force page-break to start the page numbering on a new page
\pagenumbering{arabic}

\part{Analýza zdrojových dat a návrh jejich uložení v NoSQL databázi}

\chapter{Analýza zdrojových dat}

Použitá datová sada se nachází na stránkách \href{https://mdcr.cz/Dokumenty/Verejna-doprava/Jizdni-rady,-kalendare-pro-jizdni-rady,-metodi-%281%29/Jizdni-rady-verejne-dopravy}{ministerstva dopravy}
a to konrétně \href{https://portal.cisjr.cz/pub/draha/celostatni/szdc/}{zde}. Detailní popis formátu dokumentů datové sady lze najít na stejném \href{https://portal.cisjr.cz/pub/draha/celostatni/szdc/Popis%20DJ%C5%98_CIS_v1_09.pdf}{místě}.

Datová sada se skládá z \verb|XML| souborů. Tyto soubory jsou zveřejňovaný na začátku roku pro celý rok ve složce \verb|GVD|. Dále je pak každý měsíc zveřejňována datová sada pro daný měsíc s aktualizacemi pro spoje.
Každý soubor má element \verb|CZPTTCreation|, který určuje jeho vytvoření.

\verb|XML| soubory lze rozdělit do následujících tří skupin:
\begin{itemize}
    \item Definující spoj
    \item Rušící spoj
    \item Definující náhradní spoj
\end{itemize}
Soubory definující spoje mají jako kořenový element \verb|CZPTTCISMEssage|. První důležité informace se nachází v elementu \verb|Identifiers|,
kde jsou uvedeny identifikátory pro definované spojení a vlak, který ho bude provádět. Dále element \verb|CZPTTHeader| určuje, zda spoj
přijíždí nebo pokračuje z/do zahraniční stanice. Elementy \verb|CZPTTLocation| obsahují jednotlivé stanice, kterýma vlak projiždí. Zde jsou
uvedeny i další informace ke stanici. Nejvýznamnější z nich jsou: čas příjezdu/odjezdu, typ aktivity. Po uvedení všech stanic následuje
element \verb|PlannedCalendar|, který určuje výčet dní, kdy je tento spoj prováděň.

Soubory rušící spoje mají kořenový element \verb|CZCanceledPTTMessage|. Podobně jako soubory definující spoje obsahují identifikaci spoje, který se ruší
a výčet dní, kdy se ruší. Dále už nenesou žádné informace.

Soubory definující náhradní spoje mají stejnou strukturu jako soubory definující spoje jenom s jediným rozdíle. Obsahují element
\verb|RelatedPlannedTransportIdentifiers|, který určuje, jaký spoj nahrazují. Tyto spoje mají unikátní indentifikátor vůči
normálním spojům.

\chapter{Návrh způsobu uložení dat}

\paragraph{Cíl:}
Po posouzení vlastností dat (z předchozí analýzy) a očekávaných dotazů (ze zadání) navrhnout vhodný způsob uložení dat do NoSQL databáze.
Způsob uložená musí být vhodný z hlediska způsobu nahrávání dat ze zdroje do databáze (a to i průběžného doplňování či aktualizace, bez smazání celé databáze)
a z hlediska rychlosti dotazování dat v databázi z aplikace s využitím vlastností NoSQL (s využitím klíčů a škálovatelnosti/distribuovanosti databáze).
Data lze při nahrávání ze zdroje do databáze předzpracovávat, např. kombinovat či doplňovat, odvozovat pomocná data, předpočítávat agregace, atp.
Takové předzpracování může trvat déle (kritérium vhodnosti při předzpracování v průběhu nahrávání není čas, ale vhodné využití obecných vlastností NoSQL, jako je sharding).

\paragraph{Obsah:}
Pro skupinu či každou podstatnou vlastnost dat z analýzy a dotaz ze zadání (pokud bude mít vliv na návrh) popsat,
co znamená, jaký problém představuje, jaké je řešení, proč je zvolené řešení dobré a stručně jaké jsou případné alternativy.

\paragraph{Prostředky:}
Strukturovaný text (odstavce, sekce, odrážky, atd.), kde je popsán proces získání, předzpracování a uložení dat ze zdroje do databáze.
Možno použít také pseudokód či diagramy popisující datové toky a použité struktury a vlastnosti NoSQL databází obecně.
Každé návrhové rozhodnutí musí být řádně zdůvodněno (např. části se strukturou "dotaz/vlastnost", "problém", "řešení", "důvod", "alternativy").

\paragraph{Fáze projektu:}
Po analýze dat a analýze uživatelských požadavků na aplikaci, většinou souběžně s návrhem aplikace.

\chapter{Zvolená NoSQL databáze}

\paragraph{Cíl:}
Rozhrnout a zdůvodnit jaký druh NoSQL databáze je vhodný (zdůvodnění plyne částečně již z předchozího návrhu) a jaký konkrétní produkt NoSQL databáze bude použit.

\paragraph{Obsah:}
Určit typ databáze a konkrétní produkt NoSQL, vypsat jeho vlastnosti,
které jsou pro toto řešení užitečné (a jiné než u jiných typů a produktů NoSQL)
a zdůvodnit jejich vhodnost v kontextu předchozího návrhu.

\paragraph{Prostředky:}
Stručný volný text (až několik kratších odstavců) s případným vyznačením podstatných části.

\paragraph{Fáze projektu:}
Zakončování návrhu a přechod k implementaci.

\part{Návrh, implemetace a použití aplikace}

\chapter{Návrh aplikace}

\paragraph{Cíl:}
Navrhnout hlavní části aplikace splňující požadavky zadání s důrazem na práci s na ni napojenou databází NoSQL či datovými zdroji
(při jejich předzpracování a nahrávání do NoSQL databáze).

\paragraph{Obsah:}
Použité technologie (např. skriptovací jazyk, knihovny, atp.)
a architektura (např. skript či sekvence skriptů pravidelně spouštěných v daných časových intervalech či v reakci na danou událost).
Způsob technického řešení úloh ze zadání (jejich průběh v aplikaci) a konceptů z předchozího návrhu (struktury, algoritmy, toky dat, atp.).

\paragraph{Prostředky:}
Strukturovaný text (sekce, odstavce, odrážky, atp.), případně pseudokód či obrázky, doplňující technické detaily konceptů nastíněných v předchozím návrhu.
Důraz je kladen na způsob realizace dotazů ze zadání.

\paragraph{Fáze projektu:}
Návrh aplikace a částečně po či souběžně s návrhem databáze.

\chapter{Způsob použití}

\paragraph{Cíl:}
Poskytnout stručnou dokumentaci pro zprovoznění databáze a aplikace.

\paragraph{Obsah:}
Stručně popsat, jak celé řešení zprovoznit, tj. nasadit databázi i aplikaci vč. způsobu volání aplikace (příkazový řádek, parametry) pro úlohy
předzpracování a nahrání dat ze zdroje do databáze a pro ulohy hledání nad databází tak, jak byly definovány v zadání.

\paragraph{Prostředky:}
Stručný text obsahující návod (popis) s ukázkami způsobu volání aplikace (např. pro skripty by to byl kód příkazového řadku).

\paragraph{Fáze projektu:}
Dokončování implementace, chystání dokumentace pro předání výsledného systému zákazníkovi.

\chapter{Experimenty}

\paragraph{Cíl:}
Změřit, jak aplikace a databáze fungují v praxi.

\paragraph{Obsah:}
Popis výchozí konfigurace aplikace a nasazení databáze stroje, kde budou experimenty probíhat (HW a SW).
Popis experimentů typicky představující nahrání dat ze zdroje do databáze či dotazy ze zedání s výslednými časy jejich provedení.
Případné poznámky k výsledkům experimentů.

\paragraph{Prostředky:}
Strukturvaný text, případně tabulka či graf s doprovodným textem.

\paragraph{Fáze projektu:}
Testování řešení před předáním výsledného systému zákazníkovi.

\end{document}
